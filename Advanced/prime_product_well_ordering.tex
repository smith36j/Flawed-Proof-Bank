% Author: Christian Bagshaw
% Date: August 2020
%Revised LD April 2021
% Lauren DeDieu, Jerrod M.~Smith, Kimberly Golubeva and Christian Bagshaw
% A Resource Bank for Writing Intensive Mathematics Courses
% This work is licensed under a  Creative Commons Attribution-NonCommercial-ShareAlike 4.0 International License
% http://creativecommons.org/licenses/by-nc-sa/4.0/
\section{Well-Ordering Principle}

\begin{xca}[Product of Primes]{xca:wop}
Without the use of induction, prove every integer greater than 1 can be factored as a product of primes.
\end{xca}

\begin{flaw}{flaw:wop} %change the label
We will use the well-ordering principle. Let $M$ be the set of all integers greater than one that cannot be factored as a product of primes. Assume that $M$ is not empty. Then by the well-ordering principle, there exists some least element $n \in M$. We know that $n$ is not prime, since if it was $n$ would be a product of one prime and hence would not be in $M$. So $n$ is composite. \\

This means we can write $n = xy$ for some positive integers $x$ and $y$ with $1 < x,y < n$. But, since $x < n$ this means $x$ is minimal, not $n$. We assumed that $n$ was the least element, so this is a contradiction. \\

Thus $M$ does not have a least element, so it must be empty. Therefore all integers can be factored as a product of primes.

\end{flaw}

\clearpage
\subsection{Error classification}

%Provide a brief classification and explanation of the errors in the Flawed Proof \ref{flaw:proof1}. %change the label

There is only one error
 in the Flawed Proof \ref{flaw:wop}. %change the label

 \begin{description}
    \item[F-FI]  The implication ``since $x < n$ this means $x$ is minimal, not $n$'' is false. Minimality in this case refers to the set $M$, but $x \notin M$.
 	
 \end{description}


\subsubsection{Error codes}
\begin{itemize}
	\item 	Fundamental False Implication (F-FI).
\end{itemize}
See Section \ref{sec-error} for more information about error classifications.

\clearpage
\subsection{Corrected proof}

The following is a corrected version of Flawed Proof \ref{flaw:wop}. %change the label

\begin{prf}{prf:wop} %change the label
We will use the well-ordering principle. Let $M$ be the set of all integers greater than one that cannot be factored as a product of primes. In order to derive a contradiction, assume that $M$ is not empty. Then by the well-ordering principle, there exists some least element $n \in M$. We know that $n$ is not prime, since if it was $n$ would be a product of one prime and hence would not be in $M$. So $n$ is composite. \\

This means we can write $n = xy$ for some positive integers $x$ and $y$ with $1 < x,y < n$. Since $n$ is minimal in $M$, this means $x,y \notin M$. So we can write $x$ and $y$ as a product of primes, namely we can say $x = p_1...p_k$ and $y = q_1...q_l$ for primes $p_1,...,p_k,q_1,...,q_l$, where $k,l \in \mathbb{N}$. But now this means $n = xy = p_1...p_kq_1...q_l$ so $n$ can be written as a product of primes. Thus $n $ cannot be in $M$, which is a contradiction. \\

Thus $M$ does not have a least element, so it must be empty. Therefore all integers can be factored as a product of primes.

\end{prf} 