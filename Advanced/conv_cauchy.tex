% Author: Kimberly Golubeva
% Date: 5 October 2020
%Revised LD April 2021
% Lauren DeDieu, Jerrod M.~Smith, Kimberly Golubeva and Christian Bagshaw
% A Resource Bank for Writing Intensive Mathematics Courses
% This work is licensed under a  Creative Commons Attribution-NonCommercial-ShareAlike 4.0 International License
% http://creativecommons.org/licenses/by-nc-sa/4.0/
\section{Convergent Sequences are Cauchy}

\begin{xca}{xca:conv_cauchy}
Prove that if a sequence $a_n$ converges, then $a_n$ is Cauchy.
\end{xca}

\begin{flaw}{flaw:conv_cauchy} %change the label
Suppose that $a_n \rightarrow L.$ Then
\begin{align*}
    \lim_{n,m \rightarrow \infty} (a_n - a_m) &= L - L \\
    &= 0 \\
    &< \varepsilon\;,
\end{align*}
and so $a_n$ is Cauchy.
\end{flaw}

\clearpage
\subsection{Error classification}

%Provide a brief classification and explanation of the errors in the Flawed Proof \ref{flaw:proof1}. %change the label

There is one error
% is only one error ... etc.
 in the Flawed Proof \ref{flaw:conv_cauchy}. %change the label


 \begin{description}
 	\item[EO-(C-VG):] No progress is made in this proof due to misunderstanding the definitions of Cauchy, convergence and limits.
 \end{description}


\subsubsection{Error codes}
\begin{itemize}
	\item 	Error-Caused Omission due to Content Vocabulary and Grammar (EO-(C-VG))
\end{itemize}
See Section \ref{sec-error} for more information about error classifications.

\clearpage
\subsection{Corrected proof}

The following is a corrected version of Flawed Proof \ref{flaw:conv_cauchy}. %change the label

\begin{prf}{prf:conv_cauchy} %change the label
 Suppose that $a_n \rightarrow L$. Fix $\varepsilon> 0$. Since $a_n \rightarrow L$ we know that there exists an $N \in \N$ such that $n \geq N$ implies that $|a_n - L| < \frac{\varepsilon}{2}.$ \\
 
If $m,n \geq N$ then by the triangle inequality,
\begin{align*}
    |a_n - a_m| &= |a_n-L +L - a_m| \\ &\leq |a_n-L| + |a_m-L| \\
    &< \frac{\varepsilon}{2} + \frac{\varepsilon}{2} \\
    &= \varepsilon\;.
\end{align*}
Therefore, $a_n$ is Cauchy.
\end{prf}
