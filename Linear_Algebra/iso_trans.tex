% Author: Kimberly Golubeva
% Date: 1 September 2020
% Lauren DeDieu, Jerrod M.~Smith, Kimberly Golubeva and Christian Bagshaw
% A Resource Bank for Writing Intensive Mathematics Courses
% This work is licensed under a  Creative Commons Attribution-NonCommercial-ShareAlike 4.0 International License
% http://creativecommons.org/licenses/by-nc-sa/4.0/
\section{Isomorphism of a Linear Transformation}

\begin{xca}{xca:iso_trans}
Prove that the linear map $T: P_1 \rightarrow \mathbb{R}^2$ defined by
$$T(a + bx) = \begin{pmatrix} a \\ a + b \end{pmatrix}\;,
$$
is an isomorphism and find $T^{-1}.$ (You do not need to confirm that $T$ is linear.)
\end{xca}

\begin{flaw}{flaw:iso_trans} %change the label
We prove that ker$(T) = \{ \vec{0} \}$. Suppose that $a + bx \in$ ker$(T)$. Then
$$\begin{pmatrix} a \\ a + b \end{pmatrix} = \begin{pmatrix} 0 \\ 0 \end{pmatrix}\;, $$
which implies that $a = b = 0$, and so ker$(T) = \{ \vec{0} \}.$ This means that $T$ is injective. Thus, $T$ is an isomorphism. \\

Finally, we must find $T^{-1}.$ We need $T(T^{-1})=I.$ We have
$$T(a + bx) = \begin{pmatrix} a \\ a + b \end{pmatrix}\;.$$
So then
$$T(T^{-1}) = \begin{pmatrix} a \\ a + b \end{pmatrix} = \begin{pmatrix} 1 \\ 1 \end{pmatrix}\;, $$
which means that $a=1$ and $a + b = 1.$ In particular, $a = 1$ and $b = 1-a.$
Thus,
$$T^{-1} = \begin{pmatrix} 1 \\ 1-a \end{pmatrix}\;.$$
\end{flaw}

\clearpage
\subsection{Error classification}

%Provide a brief classification and explanation of the errors in the Flawed Proof \ref{flaw:proof1}. %change the label

There are several errors
% is only one error ... etc.
 in the Flawed Proof \ref{flaw:iso_trans}. %change the label


 \begin{description}
 	\item[C-OS:] Omitted surjectivity in proving that $T$ is an isomorphism.
 	\item[EO-(C-VG):] No progress is made towards finding the inverse due to misunderstanding the definition of an inverse.
 	\item[C-FS:] The claim that $$T(T^{-1}) = \begin{pmatrix} a \\ a + b \end{pmatrix}\;,$$
 	is incorrect.
 \end{description}


\subsubsection{Error codes}
\begin{itemize}
	\item 	Content Omitted Sections (C-OS)
	\item   Error-caused Omission due to Content Vocabulary and Grammar (EO-(C-VG))
	\item   Content False Statement (C-FS)
\end{itemize}
See Section \ref{sec-error} for more information about error classifications.

\clearpage
\subsection{Corrected proof}

The following is a corrected version of Flawed Proof \ref{flaw:iso_trans}. %change the label

\begin{prf}{prf:iso_trans} %change the label
Since dim$(P_1) =$ dim$(\mathbb{R}^2) = 2$, to show $T$ is an isomorphism it suffices to show that $T$ is injective. \\

Since $T$ is linear, we know that $T$ is injective exactly when ker$(T) = \{ \vec{0} \}$. Suppose that $a + bx \in$ ker$(T)$. Then
$$T(a+bx)=\begin{pmatrix} a \\ a + b \end{pmatrix} = \begin{pmatrix} 0 \\ 0 \end{pmatrix}\;, $$
which implies that $a = b = 0$, and so ker$(T) = \{ \vec{0} \}.$ This means that $T$ is injective. Therefore, $T$ is an isomorphism. \\

\noindent Now, we must find $T^{-1}.$
Suppose that
$$T(a + bx) = \begin{pmatrix} c \\ d \end{pmatrix}\;.$$
This implies
$$\begin{pmatrix} a \\ a + b \end{pmatrix} = \begin{pmatrix} c \\ d \end{pmatrix}\;, $$
and so $a=c$ and $a + b = d.$ In particular, $a = c$ and $b = d-c.$ Then
$$T^{-1} \begin{pmatrix} c \\ d \end{pmatrix} = T^{-1}(T(a+bx))=a+bx=c+(d-c)x\;.$$
Thus,
$$T^{-1} \begin{pmatrix} c \\ d \end{pmatrix} = c+(d-c)x\;.$$
\end{prf}
