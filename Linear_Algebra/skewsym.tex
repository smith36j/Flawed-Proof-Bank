% Author: Christian Bagshaw
% Date: 21 August 2020
% Lauren DeDieu, Jerrod M.~Smith, Kimberly Golubeva and Christian Bagshaw
% A Resource Bank for Writing Intensive Mathematics Courses
% This work is licensed under a  Creative Commons Attribution-NonCommercial-ShareAlike 4.0 International License
% http://creativecommons.org/licenses/by-nc-sa/4.0/
\section{Skew-Symmetric Matrices}

\begin{xca}{xca:skewsymmat}
Decide whether the following statement is TRUE or FALSE. If it's true, then prove it. If it's false, then  find an explicit counterexample. \\

Every $2\times 2$ skew-symmetric matrix has a determinant of zero.
\end{xca}

\begin{flaw}{flaw:skewsymmat} %change the label
The statement is false. To see this, suppose we had a $2\times 2$ skew-symmetric matrix $A$. Then we can write $A = \begin{pmatrix}0 & a \\ -a & 0 \end{pmatrix}$ for some $a \in \mathbb{R}$. A simple computation gives $\det(A) = a^2$, and squares are always positive.
\end{flaw}

\clearpage
\subsection{Error classification}

%Provide a brief classification and explanation of the errors in the Flawed Proof \ref{flaw:proof1}. %change the label

There are several errors
% is only one error ... etc.
 in the Flawed Proof \ref{flaw:skewsymmat}. %change the label


 \begin{description}
 \item[WM] The student is trying to prove that the statement is always false, but the prompt says to find an explicit counterexample if the statement is false.
 \item[EO-(F-FS):] An explicit counterexample is omitted due to the false statement that $a^2$ is always positive.

 	\item[C-A:] The claim that every $2\times 2$ skew-symmetric matrix $A$ can be written in that specific form requires more justification.
 \end{description}


\subsubsection{Error codes}
\begin{itemize}
\item Wrong Method (WM)	
\item Error-caused Omission due to Fundamental False Statement (EO-(F-FS))
	\item Content Assertion (C-A)
\end{itemize}
See Section \ref{sec-error} for more information about error classifications.

\clearpage
\subsection{Corrected proof}

The following is a corrected version of Flawed Proof \ref{flaw:skewsymmat}. %change the label

\begin{prf}{prf:skewsymmat} %change the label
This statement is false. Indeed, consider $A = \begin{pmatrix}0 & 1 \\ -1 & 0 \end{pmatrix}$. This matrix $A$ is skew-symmetric since $A = -A^T$, but $\det(A) = 1 \neq 0$.
\end{prf} 