% Author: Kimberly Golubeva
% Date: 7 September 2020
% Lauren DeDieu, Jerrod M.~Smith, Kimberly Golubeva and Christian Bagshaw
% A Resource Bank for Writing Intensive Mathematics Courses
% This work is licensed under a  Creative Commons Attribution-NonCommercial-ShareAlike 4.0 International License
% http://creativecommons.org/licenses/by-nc-sa/4.0/
\section{Gram-Schmidt Procedure in a Polynomial Space}

\begin{xca}{xca:gram_sch_poly}
A basis for $P_2$ is given by $\{f,g,h\}$ where $f(x)=1,\; g(x)=x$ and $h(x)=x^2$, and the inner product is defined as $$\langle f, g \rangle = f(0)g(0) + f(1)g(1) + f(2)g(2)\;.$$ Use the Gram-Schmidt Orthogonalization Algorithm to find an orthonormal basis of $P_2.$
\end{xca}

\clearpage
\begin{flaw}{flaw:gram_sch_poly} %change the label
We start by finding $v_1, v_2$ and $v_3$ using the Gram-Schmidt process.

\begin{align*}
    v_1 &= f = 1 \\
    v_2 &= g - \frac{\langle g,v_1 \rangle}{\langle v_1, v_1 \rangle}v_1 \\
    &= x - \frac{\langle x,1 \rangle}{\langle 1, 1 \rangle} \\
    &= x - \frac{0(1) + 1(1) + 2(1)}{1+1+1}\\
    &= x - \frac{3}{3} \\
    &= x - 1 \\
    v_3 &= h - \frac{\langle h,v_1 \rangle}{\langle v_1, v_1 \rangle}v_1 - \frac{\langle h,v_2 \rangle}{\langle v_2, v_2 \rangle}v_2 \\
    &= x^2 - \frac{\langle x^2,1 \rangle}{\langle 1, 1 \rangle} - \frac{\langle x^2,x-1 \rangle}{\langle x-1, x-1 \rangle}(x-1) \\
    &= x^2 - \frac{0(1) + 1(1) + 4(1)}{3} - \frac{0(-1) + 1(0) + 4(1)}{2}(x-1)\\
    &= x^2 - \frac{5}{3} - \frac{4}{2}(x-1) \\
    &= x^2 - 2x + \frac{1}{3} \\
\end{align*}
Next, we normalize $v_1, \; v_2$ and $v_3.$
\begin{align*}
    \lVert v_1 \rVert &= \sqrt{0 + 0 + 1^2} = 1 \\
    \lVert v_2 \rVert &= \sqrt{0^2 + 1^2 + (-1)^2} = \sqrt{2} \\
    \lVert v_3 \rVert &= \sqrt{\langle v_3, v_3 \rangle} = \sqrt{1^2 + (-2)^2 + \left(\frac{1}{3}\right)^2} = \sqrt{\frac{46}{9}}\;.
\end{align*}
Then our orthonormal basis vectors are,
\begin{align*}
    u_1 &= v_1 \\
    u_2 &= \frac{1}{\sqrt{2}}v_2 \\
    u_3 &= \sqrt{\frac{9}{46}}v_3\;.
\end{align*}

\noindent Thus, $\{u_1, u_2, u_3\}$ is an orthonormal basis of $P_2.$
\end{flaw}

\clearpage
\subsection{Error classification}

%Provide a brief classification and explanation of the errors in the Flawed Proof \ref{flaw:proof1}. %change the label

There are several errors
% is only one error ... etc.
 in the Flawed Proof \ref{flaw:gram_sch_poly}. %change the label


 \begin{description}
 	\item[C-VG:] Used incorrect inner product during the normalization calculations (i.e. did not use the correct inner product definition given in the question)
 	\item[C-FS:] Incorrect computations in the normalization procedure.
 \end{description}


\subsubsection{Error codes}
\begin{itemize}
	\item 	Content Vocabulary and Grammar (C-VG)
	\item   Content False Statements (C-FS)
\end{itemize}
See Section \ref{sec-error} for more information about error classifications.

\clearpage
\subsection{Corrected proof}

The following is a corrected version of Flawed Proof \ref{flaw:gram_sch_poly}. %change the label

\begin{prf}{prf:gram_sch_poly} %change the label
We start by finding $v_1, v_2$ and $v_3$ using the Gram-Schmidt process.

\begin{align*}
    v_1 &= f = 1 \\
    v_2 &= g - \frac{\langle g,v_1 \rangle}{\langle v_1, v_1 \rangle}v_1 \\
    &= x - \frac{\langle x,1 \rangle}{\langle 1, 1 \rangle} \\
    &= x - \frac{0(1) + 1(1) + 2(1)}{1+1+1}\\
    &= x - \frac{3}{3} \\
    &= x - 1 \\
    v_3 &= h - \frac{\langle h,v_1 \rangle}{\langle v_1, v_1 \rangle}v_1 - \frac{\langle h,v_2 \rangle}{\langle v_2, v_2 \rangle}v_2 \\
    &= x^2 - \frac{\langle x^2,1 \rangle}{\langle 1, 1 \rangle} - \frac{\langle x^2,x-1 \rangle}{\langle x-1, x-1 \rangle}(x-1) \\
    &= x^2 - \frac{0(1) + 1(1) + 4(1)}{3} - \frac{0(-1) + 1(0) + 4(1)}{2}(x-1)\\
    &= x^2 - \frac{5}{3} - \frac{4}{2}(x-1) \\
    &= x^2 - 2x + \frac{1}{3} \\
\end{align*}
Next, we normalize $v_1, \; v_2$ and $v_3.$
\begin{align*}
    \lVert v_1 \rVert &= \sqrt{\langle v_1, v_1 \rangle} = \sqrt{1^2+1^2+1^2} = \sqrt{3} \\
    \lVert v_2 \rVert &= \sqrt{\langle v_2, v_2 \rangle} = \sqrt{-1^2+0^2+1^2} = \sqrt{2} \\
    \lVert v_3 \rVert &= \sqrt{\langle v_3, v_3 \rangle} = \sqrt{\left(\frac{1}{3}\right)^2 + \left(\frac{-2}{3}\right)^2 + \left(\frac{1}{3}\right)^2} = \sqrt{\frac{2}{3}}\;.
\end{align*}
Then our orthonormal basis vectors are,
\begin{align*}
    u_1 &= \frac{1}{\sqrt{3}}v_1 \\
    u_2 &= \frac{1}{\sqrt{2}}v_2 \\
    u_3 &= \sqrt{\frac{3}{2}}v_3\;.
\end{align*}

\noindent Thus, $\{u_1, u_2, u_3\}$ is an orthonormal basis of $P_2.$

\end{prf}
