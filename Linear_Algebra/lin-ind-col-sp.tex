% Author: Jerrod Smith
% Date: 23 June 2020
% Lauren DeDieu, Jerrod M.~Smith, Kimberly Golubeva and Christian Bagshaw
% A Resource Bank for Writing Intensive Mathematics Courses
% This work is licensed under a  Creative Commons Attribution-NonCommercial-ShareAlike 4.0 International License
% http://creativecommons.org/licenses/by-nc-sa/4.0/
\section{Linear Independence}

Let $\underline{\bm{0}}$ denote the $m\times n$ zero matrix.

\begin{xca}{xca:lin-ind-col-sp}
Let $C$ be a nonzero $m\times n$ matrix and let $\bm{v}_1, \bm{v}_2,\ldots \bm{v}_\ell$ be nonzero vectors in $\R^n$.  Prove that if $\{C\bm{v}_1,C\bm{v}_2,\ldots, C\bm{v}_\ell\}$ is linearly independent, then $\{\bm{v}_1, \bm{v}_2,\ldots \bm{v}_\ell\}$ is linearly independent.
\end{xca}

\begin{flaw}{flaw:lin-ind-col-sp}
Suppose $\{C\bm{v}_1,C\bm{v}_2,\ldots, C\bm{v}_\ell\}$ is linearly independent. Then
	\[a_1(C\bm{v}_1) + a_2(C\bm{v}_2) + \ldots + a_\ell (C\bm{v}_\ell) = \vec{0}_m\]
	implies that $
	a_1 = a_2 = \ldots = a_\ell = 0$.
	Now, we have that \[	\vec{0}_m = a_1(C\bm{v}_1) + a_2(C\bm{v}_2) + \ldots + a_\ell (C\bm{v}_\ell)
		 = C(a_1\bm{v}_1 + a_2\bm{v}_2 + \ldots + a_\ell\bm{v}_\ell) \]
and, since $C\neq \underline{\bm{0}}$, it follows that $\vec{0}_n = a_1\bm{v}_1 + a_2\bm{v}_2 + \ldots + a_\ell\bm{v}_\ell$.
We know that $a_1 = a_2 = \ldots = a_\ell = 0$.  This implies that $\{\bm{v}_1, \bm{v}_2,\ldots \bm{v}_\ell\}$ is linearly independent.
\end{flaw}

\clearpage
\subsection{Error classification}

There are several errors in the Flawed Proof \ref{flaw:lin-ind-col-sp}.

\begin{description}
	\item[F-Log:] The Flawed Proof \ref{flaw:lin-ind-col-sp} incorrectly begins by considering a linear combination of the vectors  $C\bm{v}_1,C\bm{v}_2,\ldots, C\bm{v}_\ell$.  In order to prove that the set $\{\bm{v}_1, \bm{v}_2,\ldots \bm{v}_\ell\}$ is linearly independent, we must prove that: if $\vec{0}_n = a_1\bm{v}_1 + a_2\bm{v}_2 + \ldots + a_\ell\bm{v}_\ell$ for some scalars $a_i \in \R$, $1\leq i \leq \ell$, then $a_i = 0$ for all $1\leq i \leq \ell$.  In order to prove this statement, we must begin with the assumption that ``$\vec{0}_n = a_1\bm{v}_1 + a_2\bm{v}_2 + \ldots + a_\ell\bm{v}_\ell$ for some scalars $a_i \in \R$, $1 \leq i \leq \ell$".
		\item[N-O:] The coefficients $a_1, a_2, \ldots, a_\ell$ are undefined.

	\item[C-FI:] The implication ``since $C\neq \underline{\bm{0}}$, it follows that $\vec{0}_n = a_1\bm{v}_1 + a_2\bm{v}_2 + \ldots + a_\ell\bm{v}_\ell$" is false.  In this setting, the claim that $\vec{0}_n = a_1\bm{v}_1 + a_2\bm{v}_2 + \ldots + a_\ell\bm{v}_\ell$ is equivalent to stating that $\nullsp(C) = \{\vec{0}_n\}$. But the fact that $C$ is a nonzero matrix does not imply that the nullspace of $C$ is equal to $\{\vec{0}_n\}$. For example, the nonzero matrix $C = \begin{bmatrix} 1 & 0 \\ 0 & 0 \end{bmatrix}$ has nullspace $\nullsp(C) = \spn \left \{ \begin{bmatrix} 0 \\ 1 \end{bmatrix} \right \}$.
	\item[C-FI:] The implication ``We know that $a_1 = a_2 = \ldots = a_\ell = 0$.  This implies that $\{\bm{v}_1, \bm{v}_2,\ldots \bm{v}_\ell\}$ is linearly independent." is false.  The fact that $\vec{0}_n = 0\bm{v}_1 + 0\bm{v}_2 + \ldots + 0\bm{v}_\ell$ does not imply that $\{\bm{v}_1, \bm{v}_2,\ldots \bm{v}_\ell\}$ is linearly independent. To prove that $\{\bm{v}_1, \bm{v}_2,\ldots \bm{v}_\ell\}$ is linearly independent, one must show that $a_1 = a_2 = \ldots = a_\ell = 0$ is the \textbf{only solution} to the equation $\vec{0}_n = a_1\bm{v}_1 + a_2\bm{v}_2 + \ldots + a_\ell\bm{v}_\ell$, where $a_i \in \R$, $1\leq i \leq \ell$.
\end{description}

\subsubsection{Error codes}
\begin{itemize}
	\item 	Fundamental Logical Order (F-Log)
	\item 	Novice Local Omission (N-O)
	\item   Content False Implication (C-FI)
\end{itemize}
See Section \ref{sec-error} for more information about error classifications.


\clearpage
\subsection{Corrected proof}

The following is a corrected version of Flawed Proof \ref{flaw:lin-ind-col-sp}.
\begin{prf}{prf:lin-ind-col-sp}
Let $C$ be a nonzero $m\times n$ matrix and let $\bm{v}_1, \bm{v}_2,\ldots \bm{v}_\ell$ be nonzero vectors in $\R^n$.
Suppose that $\{C\bm{v}_1,C\bm{v}_2,\ldots, C\bm{v}_\ell\}$ is linearly independent.
Suppose that
\[
\vec{0}_n = a_1\bm{v}_1 + a_2\bm{v}_2 + \ldots + a_\ell\bm{v}_\ell
\]
for some scalars $a_i \in \R$, $1 \leq i \leq \ell$.  Multiplying this equation by the $m\times n$ matrix $C$ we obtain
\begin{align*}
	\vec{0}_m  & = C\, \vec{0}_n\\
	& = C(a_1\bm{v}_1 + a_2\bm{v}_2 + \ldots + a_\ell\bm{v}_\ell) \\
& = a_1(C\bm{v}_1) + a_2(C\bm{v}_2) + \ldots + a_\ell (C\bm{v}_\ell).
\end{align*}
Since $\{C\bm{v}_1,C\bm{v}_2,\ldots, C\bm{v}_\ell\}$ is linearly independent, it must be the case that $a_i = 0$ for all $1\leq i \leq \ell$.  Thus, the set $\{\bm{v}_1, \bm{v}_2,\ldots \bm{v}_\ell\}$ is linearly independent.
\end{prf} 