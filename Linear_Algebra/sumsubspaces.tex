%Author: Kimberly Golubeva
%Date: July 14 2020
% Lauren DeDieu, Jerrod M.~Smith, Kimberly Golubeva and Christian Bagshaw
% A Resource Bank for Writing Intensive Mathematics Courses
% This work is licensed under a  Creative Commons Attribution-NonCommercial-ShareAlike 4.0 International License
% http://creativecommons.org/licenses/by-nc-sa/4.0/
\section{Sum of Subspaces}

\begin{xca}[]{xca:sumsubspaces}
Prove or disprove the following statement. If $V$ and $W$ are both subspaces of $\mathbb{R}^n$, then $V+W$ is also a subspace of $\mathbb{R}^n$.
\end{xca}

\begin{flaw}{flaw:sumsubspaces}
This statement is true. We will use the definition of a subspace to prove that $V+W$ is a subspace.

\begin{enumerate}
\item Contains $\vec{0}$: Take $\vec{0} \in V$ and $\vec{0} \in W$. Then $\vec{0} + \vec{0} = \vec{0} \in V + W.$
\item Closed Under Addition: Let $\vec{v} \in V$ and $\vec{w} \in W$. Then $\vec{v} + \vec{w} \in V+W$, which means that $V+W \in \mathbb{R}^n.$
\item Closed Under Scalar Multiplication: Let  $\vec{v} \in V+W$ and $w \in \mathbb{R}.$ Then $\vec{v}(w)= \vec{v}w \in V+W.$
\end{enumerate}
\end{flaw}

\clearpage
\subsection{Error classification}

%Provide a brief classification and explanation of the errors in the Flawed Proof \ref{flaw:proof1}. %change the label

There are several errors
% is only one error ... etc.
 in the Flawed Proof \ref{flaw:sumsubspaces}.

\begin{description}
 	\item[EO-(C-VG):] Misunderstanding what it means to be `closed' under addition and scalar multiplication leading to omission of major parts of the proof. Start with elements that are in $V+W$ and show that they are still in $V+W$ after these operations are performed.
 %	\item [F-A:] The `Closed Under Addition' and `Closed Under Scalar Multiplication' portions of the proof are asserted due to misunderstanding the notion of being `closed'. 	
 	\item[N-N:] To avoid confusing notation, choose a different letter for the scalar.

 \end{description}


\subsubsection{Error codes}
\begin{itemize}
	\item 	Error-Caused Omission due to Content Vocabulary $\&$ Grammar (EO-(C-VG))
	%\item   Fundamental Assertion (F-A)
	\item   Novice Notation (N-N)
\end{itemize}
See Section \ref{sec-error} for more information about error classifications.

\clearpage
\subsection{Corrected proof}

The following is a corrected version of Flawed Proof \ref{flaw:sumsubspaces}. %change the label

\begin{prf}{prf:sumsubspaces} %change the label
This statement is true. We will use the definition of a subspace to prove that $V+W$ is a subspace.

\begin{enumerate}
\item Contains $\vec{0}$: Since both $V$ and $W$ are subspaces, take $\vec{0} \in V$ and $\vec{0} \in W$. Then $\vec{0} + \vec{0} = \vec{0} \in V + W.$
\item Closed Under Addition: Let $\vec{x}, \vec{y} \in V+W$. By definition, there exists $\vec{v_1}, \vec{v_2} \in V$ and $\vec{w_1}, \vec{w_2} \in W$ such that $\vec{x} = \vec{v_1} +\vec{w_1}$ and $\vec{y} = \vec{v_2} +\vec{w_2}$. Then
$$\vec{x} + \vec{y} = (\vec{v_1} +\vec{w_1}) + (\vec{v_2} +\vec{w_2}) = (\vec{v_1} +\vec{v_2}) + (\vec{w_1} +\vec{w_2}).$$
Since $V$ is a subspace, it is closed under addition and so $(\vec{v_1} +\vec{v_2}) \in V$. Similarly, $(\vec{w_1} +\vec{w_2}) \in W$. This means that $(\vec{v_1} +\vec{v_2}) + (\vec{w_1} +\vec{w_2}) \in V + W$. Thus, $\vec{x} + \vec{y} \in V+W$ and so $V+W$ is closed under addition.
\item Closed Under Scalar Multiplication: Let  $\vec{x} \in V+W$ and $k \in \mathbb{R}.$ By definition, there exists $\vec{v} \in V$ and $\vec{w} \in W$ such that $\vec{x} = \vec{v} +\vec{w}$. Then
$$k\vec{x} = k(\vec{v} +\vec{w})= k\vec{v} + k\vec{w}.$$
Since $V$ is a subspace, it is closed under scalar multiplication and so $k\vec{v} \in V$. Similarly, $k\vec{w} \in W$. This means that $k\vec{v} + k\vec{w} \in V+W$. Thus, $k\vec{x} \in V+W$ and so $V+W$ is closed under scalar multiplication.
\end{enumerate}
\end{prf}
