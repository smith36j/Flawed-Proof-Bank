% Author: Kimberly Golubeva
% Date: 21 August 2020
% Lauren DeDieu, Jerrod M.~Smith, Kimberly Golubeva and Christian Bagshaw
% A Resource Bank for Writing Intensive Mathematics Courses
% This work is licensed under a  Creative Commons Attribution-NonCommercial-ShareAlike 4.0 International License
% http://creativecommons.org/licenses/by-nc-sa/4.0/
\section{Basis of a Polynomial Space }

\begin{xca}{xca:basis_span_poly}
Is $S = \{ x^3 + 2, \; x^2-3x+1, \; 5 \}$ a basis of $U =\text{span}(S)?$
\end{xca}

\begin{flaw}{flaw:basis_span_poly} %change the label
Since $U$ is the span of $S$, then by definition of a basis, $S$ is trivially a basis of $U.$
\end{flaw}

\clearpage
\subsection{Error classification}

%Provide a brief classification and explanation of the errors in the Flawed Proof \ref{flaw:proof1}. %change the label

There is one error
% is only one error ... etc.
 in the Flawed Proof \ref{flaw:basis_span_poly}. %change the label


 \begin{description}
 	\item[EO-(C-VG):] Misunderstanding the definition of basis caused omission of the linear independence section of the proof. A basis $S$ of a vector space $U$ must span $U$ and be linearly independent.
 \end{description}


\subsubsection{Error codes}
\begin{itemize}
	\item 	Error-caused Omission due to Content Vocabulary and Grammar (EO-(C-VG))
	%\item   Content Omitted Sections (C-OS)
\end{itemize}
See Section \ref{sec-error} for more information about error classifications.

\clearpage
\subsection{Corrected proof}

The following is a corrected version of Flawed Proof \ref{flaw:basis_span_poly}. %change the label

\begin{prf}{prf:basis_span_poly} %change the label
In order for $S$ to be a basis of $U$, it must satisfy two conditions: \\

First, $S$ must span $U$. In this case, since we have that $U =\text{span}(S),$ we know that $S$ spans $U.$ \\

Secondly, the set $S$ must be linearly independent. Since the three polynomials in $S$ have different degrees, this implies that they are linearly independent. \\

Thus, since $S$ spans $U$ and is linearly independent, we can conclude that $S$ is a basis of $U.$
\end{prf}
