% Author: Kimberly Golubeva
% Date: 20 April 2020
% Lauren DeDieu, Jerrod M.~Smith, Kimberly Golubeva and Christian Bagshaw
% A Resource Bank for Writing Intensive Mathematics Courses
% This work is licensed under a  Creative Commons Attribution-NonCommercial-ShareAlike 4.0 International License
% http://creativecommons.org/licenses/by-nc-sa/4.0/
\section{Surjection}

\begin{xca}{xca:surjection_backwards}
Consider the function $f: \mathbb{R}- \{0\} \rightarrow \mathbb{R}-\{1\}$ defined by 
$$f(x) = \frac{x + 15}{x}\;.$$
Is $f$ surjective? Prove or disprove.
\end{xca}

\begin{flaw}{flaw:surjection_backwards} %change the label
Yes, $f$ is surjective since for all $y \in \mathbb{R}-\{1\}$, there exists an $x \in \mathbb{R}-\{0\}$ such that $y=f(x).$ Namely, 
\begin{align*}
    y =& f(x) \\
    y =& \frac{x+15}{x} \\
    yx =& x+15 \\
    yx - x =& 15 \\
    x(y-1) =& 15 \\
    x =& \frac{15}{y-1}\;.
\end{align*}
Thus, we can see that $f$ is surjective.
\end{flaw}

\clearpage
\subsection{Error classification}

%Provide a brief classification and explanation of the errors in the Flawed Proof \ref{flaw:proof1}. %change the label

There are several errors
% is only one error ... etc.
 in the Flawed Proof \ref{flaw:surjection_backwards}. %change the label

 
 \begin{description}
    \item[EO-F-A: ] Although the method of finding $x$ is correct, the entirety of the proof is omitted (that is, verifying that $f(x)=y$ with the particular $x$ that was calculated) due to the assumption that $y=f(x)$. 
 	\item[C-VG: ] Misunderstanding the definition of surjective. 
 \end{description}

 
\subsubsection{Error codes}
\begin{itemize}
	\item 	Error-Caused Omission Fundamental Assertion (EO-F-A)
	\item   Content Vocabulary and Grammar
\end{itemize}
See Section \ref{sec-error} for more information about error classifications.

\clearpage
\subsection{Corrected proof}

The following is a corrected version of Flawed Proof \ref{flaw:surjection_backwards}. %change the label

\begin{prf}{prf:surjection_backwards} %change the label
Yes, $f$ is surjective. To prove this, we begin by finding a suitable $x \in \mathbb{R}-\{0\}$:
\begin{align*}
    y =& f(x) \\
    y =& \frac{x+15}{x} \\
    yx =& x+15 \\
    yx - x =& 15 \\
    x(y-1) =& 15 \\
    x =& \frac{15}{y-1}\;.
\end{align*}
So choose $x= \frac{15}{y-1}$ with $x \in \mathbb{R}-\{0\}$. Then,

$$f \left (\frac{15}{y-1} \right ) = \frac{\frac{15}{y-1} + 15}{\frac{15}{y-1}} = \frac{15 + 15(y-1)}{15} = \frac{15+15y-15}{15} = y\;. $$

Thus, we can see that $f$ is surjective since for all $y \in \mathbb{R}-\{1\}$, there exists an $x \in \mathbb{R}-\{0\}$ such that $f(x)=y.$  
\end{prf}