% Author: Christian Bagshaw
% Date: DD MMM 2020
% Lauren DeDieu, Jerrod M.~Smith, Kimberly Golubeva and Christian Bagshaw
% A Resource Bank for Writing Intensive Mathematics Courses
% This work is licensed under a  Creative Commons Attribution-NonCommercial-ShareAlike 4.0 International License
% http://creativecommons.org/licenses/by-nc-sa/4.0/
\section{Equivalence Relations}

\begin{xca}[Modular Congruence is Transitive]{xca:equiv_modular}
Prove that modular congruence is transitive; that is, for integers $a,b,c,n$, prove that if $a \equiv b \mod n$ and $b \equiv c \mod n$, then $a \equiv c \mod n$. 

\end{xca}

\begin{flaw}{flaw:equiv_modular} %change the label

Since $a \equiv b \mod n$, then $a - b = nk$. Now if we let $n = c-b$ and $k=1$, then $a-b = c-b = nk$, or $a-c = nk$. So $a\equiv c \mod n$. 
\end{flaw}

\clearpage
\subsection{Error classification}

%Provide a brief classification and explanation of the errors in the Flawed Proof \ref{flaw:proof1}. %change the label

There are several errors
% is only one error ... etc.
 in the Flawed Proof \ref{flaw:equiv_modular}. 
 
 \begin{description}
    \item[N-N] $k$ is not defined. 
    \item[C-Eg] The statement needs to be proven for all $n$, but the proof attempts to prove the statement for a specific $n$
    \item[F-FI] That $a-b = c-b = nk$ implies $a-c = nk$ is false. 

 	
 \end{description}

 
\subsubsection{Error codes}
\begin{itemize}
    \item Novice Notation (N-N)
	\item Content Proof by Example (C-Eg)
	\item Fundamental False Implication (F-FI)
\end{itemize}
See Section \ref{sec-error} for more information about error classifications.

\clearpage
\subsection{Corrected proof}

The following is a corrected version of Flawed Proof \ref{flaw:equiv_modular}. %change the label

\begin{prf}{prf:equiv_modular} %change the label
For integers $a,b,c,n$ suppose that $a \equiv b \mod n$ and $b \equiv c \mod n$. This means there exists integer $k, l$ such that $a-b = nk$ and $b-c = nl$. Solving for $b$ in the second equation gives $b = nl + c$. Substituting into the second equation means $a - nl - c = nk$, so $a-c = nl + nk = n(l + k)$. Thus $n$ divides $a-c$, so by definition $a\equiv c \mod n$. 
\end{prf}