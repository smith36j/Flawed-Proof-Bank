% Author: Kimberly Golubeva
% Date: 28 September 2020
% Lauren DeDieu, Jerrod M.~Smith, Kimberly Golubeva and Christian Bagshaw
% A Resource Bank for Writing Intensive Mathematics Courses
% This work is licensed under a  Creative Commons Attribution-NonCommercial-ShareAlike 4.0 International License
% http://creativecommons.org/licenses/by-nc-sa/4.0/
\section{Addition and Multiplication of Congruence Relations}

\begin{xca}{xca:cong_add_mult}
Prove that if $a \equiv b\;(mod\;n)$ and $c \equiv d \;(mod\;n)$, then
\begin{enumerate}
    \item $(a+c) \equiv (b+d) \;(mod\;n)$\;,
    \item $ac \equiv bd \;(mod\;n)$\;.
\end{enumerate}
\end{xca}

\begin{flaw}{flaw:cong_add_mult} %change the label
Suppose that $a \equiv b\;(mod\;n)$ and $c \equiv d \;(mod\;n)$\;. By the definition of congruence, there exists integers $m$ and $k$ such that 
$$a-b = mn \quad \text{and} \quad c-d = kn\;.$$

We have
\begin{align*}
    (a+c) - (b+d) &= n(m+k) \\
    a-b+c-d &= n(m+k) \\
\end{align*}
which implies that $(a+c) \equiv (b+d)\; mod\;(n)\;.$ \\

We have
\begin{align*}
    ac - bd &=n(cm + bk), \\
    &= c(a-b) + b(c-d) \\
\end{align*}
which implies that $ac \equiv bd \;(mod\;n)$\;.

\end{flaw}

\clearpage
\subsection{Error classification}

%Provide a brief classification and explanation of the errors in the Flawed Proof \ref{flaw:proof1}. %change the label

There are several errors
% is only one error ... etc.
 in the Flawed Proof \ref{flaw:cong_add_mult}. %change the label

 
 \begin{description}
 	\item[C-Cir:] Both portions of the proof use the conclusion in the course of the proof. In particular, the conclusions $(a+c) - (b+d) = n(m+k)$ and $ac - bd =n(cm + bk)$ are assumed at the beginning of the proof. 
 	\item[C-EO-Cir:] Many of the important justification steps are omitted due to the arguments being circular. 
 	\item[F-A:] The results are asserted as there are no justifications for the conclusions reached. 
 \end{description}

 
\subsubsection{Error codes}
\begin{itemize}
	\item 	Content Circular Argument (C-Cir)
	\item   Content Error-Caused Omission due to Circular Argument (C-EO-Cir)
	\item   Fundamental Assertion (F-A)
\end{itemize}
See Section \ref{sec-error} for more information about error classifications.

\clearpage
\subsection{Corrected proof}

The following is a corrected version of Flawed Proof \ref{flaw:cong_add_mult}. %change the label

\begin{prf}{prf:cong_add_mult} %change the label
Suppose that $a \equiv b\;(mod\;n)$ and $c \equiv d \;(mod\;n)$\;. By the definition of congruence, there exists integers $m$ and $k$ such that 
$$a-b = mn \quad \text{and} \quad c-d = kn\;.$$

\noindent First we prove that  $(a+c) \equiv (b+d) \;(mod\;n)$\;.\\

We have
\begin{align*}
    a-b+c-d &= n(m+k) \\
    (a+c) - (b+d) &= n(m+k)\;,
\end{align*}
and so $n \vert (a+c) - (b+d)$ which implies that     $(a+c) \equiv (b+d)\; mod\;(n)\;.$ \\

\noindent Next, we prove that $ac \equiv bd \;(mod\;n)$\;. Here, we use that fact that $-bc + bc = 0.$\\

We have
\begin{align*}
    ac - bd &= ac + 0 -bd \\
    &= ac + (-bc + bc) - bd \\
    &= c(a-b) + b(c-d) \\
    &=c(mn) + b(kn) \\
    &=n(cm + bk)\;, 
\end{align*}
and so $n \vert (ac-bd)$ which implies that $ac \equiv bd \;(mod\;n)$\;.

\end{prf}
