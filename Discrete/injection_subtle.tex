% Author: Kimberly Golubeva
% Date: 20 April 2020
% Lauren DeDieu, Jerrod M.~Smith, Kimberly Golubeva and Christian Bagshaw
% A Resource Bank for Writing Intensive Mathematics Courses
% This work is licensed under a  Creative Commons Attribution-NonCommercial-ShareAlike 4.0 International License
% http://creativecommons.org/licenses/by-nc-sa/4.0/
\section{Injection}

\begin{xca}{xca:injection_subtle}
Consider $f: \mathbb{R} \rightarrow \mathbb{R}$ and $g: \mathbb{Z} \rightarrow \mathbb{Z}$ where $f(x) = 3x-4$ and $g(n) = n^4$ for all $n \in \mathbb{Z}.$ Is $f$ injective? Is $g$ injective? Prove or disprove.
\end{xca}

\begin{flaw}{flaw:injection_subtle} %change the label
Yes, $f$ is injective. Suppose that $a, b \in \mathbb{R}$ and that $f(a) = f(b)$. Then:
\begin{align*}
    f(a) &= f(b) \\
    3a-4 &= 3b-4 \\
    3a &= 3b \\
    a &= b\;.
\end{align*}
Thus, $f$ is injective. 

Yes, $g$ is injective. Suppose that $a, b \in \mathbb{Z}$ and that $g(a) = g(b)$. Then:
\begin{align*}
    g(a) &= g(b) \\
    \sqrt[4]{a^4} &= \sqrt[4]{b^4} \\
    a &= b \;.
\end{align*}
Thus, $g$ is injective. 
\end{flaw}

\clearpage
\subsection{Error classification}

%Provide a brief classification and explanation of the errors in the Flawed Proof \ref{flaw:proof1}. %change the label

There is only one error
 in the Flawed Proof \ref{flaw:injection_subtle}. %change the label

 
 \begin{description}
 	\item[EO-C-FS] The entirely of the proof was omitted due to the false statement that $\sqrt[4]{a^4} = \sqrt[4]{b^4} \implies
    a = b.$ Namely, it was assumed that $a$ and $b$	are positive.
 \end{description}

 
\subsubsection{Error codes}
\begin{itemize}
	\item 	Error-cased Omission Content False Statement
\end{itemize}
See Section \ref{sec-error} for more information about error classifications.

\clearpage
\subsection{Corrected proof}

The following is a corrected version of Flawed Proof \ref{flaw:injection_subtle}. %change the label

\begin{prf}{prf:injection_subtle} %change the label
Yes, $f$ is injective. Suppose that $a, b \in \mathbb{R}$ and that $f(a) = f(b)$. Then:
\begin{align*}
    f(a) &= f(b) \\
    3a-4 &= 3b-4 \\
    3a &= 3b \\
    a &= b\;.
\end{align*}
Thus, $f$ is injective. 

No, $g$ is not injective. To prove this, we will consider the following counterexample. Let $a = 2$ and $b=-2$. Then:
\begin{align*}
    g(a) &= g(b) \\
    16 &= 16 \\
    2^4 &= (-2)^4\;,
\end{align*}
and so $g(a)=g(b)=16,$ but $a=2 \neq -2 = b$. Thus, $g$ is not injective. 
\end{prf}
