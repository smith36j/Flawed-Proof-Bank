% Lauren DeDieu, Jerrod M.~Smith, Kimberly Golubeva and Christian Bagshaw
% A Resource Bank for Writing Intensive Mathematics Courses
% This work is licensed under a  Creative Commons Attribution-NonCommercial-ShareAlike 4.0 International License
% http://creativecommons.org/licenses/by-nc-sa/4.0/
\section{Online Discussion Board Activity}
\textbf{Learning Outcomes}
\begin{enumerate}
	\item Create proofs of mathematical statements; establish clear and consistent notation; write a clear and organized logical argument. 
	\item Work with precise definitions and reason in an abstract setting.
	\item Verify that an abstract mathematical object satisfies a given definition, or is a counterexample. 
	\item Read and critique a mathematical proof, and use a rubric to provide constructive feedback to a peer.
\end{enumerate}

\noindent\textbf{Instructions}
\begin{enumerate}
	\item Consider the \textbf{Flawed Proof} the (true) statement below.  \emph{The proof contains both mathematical and writing errors}.
	\item Use the Peer Evaluation Rubric ($\S$\ref{sec-peer}) for Mathematical Writing to score the proof in four categories: Notation, Language \& Clarity, Logic, and Completeness.  
	\item Briefly (300-400 words) describe the mathematical and writing errors that you find.
	\item Read the evaluations given by your group members.  Did you give the proof the same scores? Did you find the all the same errors? o	What (if any) are the differences between your evaluations? Post replies (50-100 words) to each of your group members discussing your observations.
	\item[***] \textbf{Remember to assume positive intent; keep your comments constructive and professional.}  
\end{enumerate}

\vfill

\noindent\textbf{Statement}: \emph{Statement text}.

\begin{proof}[Flawed Proof]
\emph{Proof text}.
\end{proof}

