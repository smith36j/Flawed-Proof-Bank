% Lauren DeDieu, Jerrod M.~Smith, Kimberly Golubeva and Christian Bagshaw
% A Resource Bank for Writing Intensive Mathematics Courses
% This work is licensed under a  Creative Commons Attribution-NonCommercial-ShareAlike 4.0 International License
% http://creativecommons.org/licenses/by-nc-sa/4.0/
\section{Peer Evaluation Activity}
\textbf{Learning Outcomes}
\begin{enumerate}
	\item Create proofs of mathematical statements; establish clear and consistent notation; write a clear and organized logical argument. 
	\item Work with precise definitions and reason in an abstract setting.
	\item Verify that an abstract mathematical object satisfies a given definition, or is a counterexample. 
	\item Read and critique a mathematical proof, and use a rubric to provide constructive feedback to a peer.
	\item[***] \textbf{To achieve these outcomes, and to receive valuable feedback on your writing, is important for you to write your proof on your own, without aids, and without discussing the proof with your peers in advance.} 
\end{enumerate}

\noindent\textbf{Instructions}
\begin{enumerate}
	\item Formulate a \underline{precise statement} of, and write a \underline{complete and detailed proof} of, the following statement (below).
	\item Post your proof to the Writing Assignment \# N discussion board; you can add your proof as a PDF attachment or type it directly using the \LaTeX\, environment.
	\item Read the proofs written by your group members and use the \textbf{Peer Evaluation Rubric for Mathematical Writing} ($\S$\ref{sec-peer}) to score your peers proofs as ``Beginning, Developing" or ``Accomplished" in four categories: Notation, Language \& Clarity, Logic, and Completeness.
	\item You will also provide brief written comments to explain your scoring.
	\item[***] \textbf{Remember to assume positive intent; keep your comments constructive and professional.}  
\end{enumerate}

\noindent\textbf{Statement}

If the composition of two functions is surjective and the second function is injective, then the first function must be surjective.

\noindent\emph{Instructor note: try to keep the statement general so that students have to formulate it precisely and introduce their own notation}.

