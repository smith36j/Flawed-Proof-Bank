% Christian Bagshaw, Lauren DeDieu, Kimberly Golubeva, and Jerrod M.~Smith 
% A Resource Bank for Writing Intensive Mathematics Courses
% This work is licensed under a  Creative Commons Attribution-NonCommercial-ShareAlike 4.0 International License
% http://creativecommons.org/licenses/by-nc-sa/4.0/
\usepackage[framemethod=TikZ]{mdframed}
\usepackage{amsthm}
\usepackage{amsmath}
\usepackage{amssymb}
\usepackage{verbatim} %% for command \begin{comment} \end{comment}
\usepackage{bm} % bold math
\usepackage{mathtools}
\usepackage{xcolor} % to define colours
\usepackage{todonotes}
\usepackage{pdfpages} %% including whole pages pdf files
\usepackage{multicol}
%\usepackage{draftwatermark}
%	\SetWatermarkText{Draft. Date.}
%	\SetWatermarkScale{4}
\usepackage[pdftex,colorlinks,linktocpage=true,citecolor=blue,linkcolor=magenta]{hyperref} 
%%%%%%%%%%%%%%%%%%%%%%%%%%%%%%
%		Coloured Box Environments
%%%%%%%%%%%%%%%%%%%%%%%%%%%%%%
% Theorem
\newcounter{thm}[section] \setcounter{thm}{0}
\renewcommand{\thethm}{\arabic{chapter}.\arabic{section}.\arabic{thm}}
\newenvironment{thm}[2][]{%
\refstepcounter{thm}%
\ifstrempty{#1}%
{\mdfsetup{%
frametitle={%
\tikz[baseline=(current bounding box.east),outer sep=0pt]
\node[anchor=east,rectangle,fill=cyan!30]
{\strut Theorem~\thethm};}}
}%
{\mdfsetup{%
frametitle={%
\tikz[baseline=(current bounding box.east),outer sep=0pt]
\node[anchor=east,rectangle,fill=cyan!30]
{\strut Theorem~\thethm:~#1};}}%
}%
\mdfsetup{innertopmargin=10pt,linecolor=cyan!30,%
linewidth=2pt,topline=true,%
frametitleaboveskip=\dimexpr-\ht\strutbox\relax
}
\begin{mdframed}[]\relax%
\label{#2}}{\end{mdframed}}
%%%%%%%%%%%%%%%%%%%%%%%%%%%%%%
% Lemma
\newenvironment{lem}[2][]{%
\refstepcounter{thm}%
\ifstrempty{#1}%
{\mdfsetup{%
frametitle={%
\tikz[baseline=(current bounding box.east),outer sep=0pt]
\node[anchor=east,rectangle,fill=olive!30]
{\strut Lemma~\thethm};}}
}%
{\mdfsetup{%
frametitle={%
\tikz[baseline=(current bounding box.east),outer sep=0pt]
\node[anchor=east,rectangle,fill=olive!30]
{\strut Lemma~\thethm:~#1};}}%
}%
\mdfsetup{innertopmargin=10pt,linecolor=olive!30,%
linewidth=2pt,topline=true,%
frametitleaboveskip=\dimexpr-\ht\strutbox\relax
}
\begin{mdframed}[]\relax%
\label{#2}}{\end{mdframed}}
%%%%%%%%%%%%%%%%%%%%%%%%%%%%%%
% Corollary
\newenvironment{cor}[2][]{%
\refstepcounter{thm}%
\ifstrempty{#1}%
{\mdfsetup{%
frametitle={%
\tikz[baseline=(current bounding box.east),outer sep=0pt]
\node[anchor=east,rectangle,fill=violet!30]
{\strut Corollary~\thethm};}}
}%
{\mdfsetup{%
frametitle={%
\tikz[baseline=(current bounding box.east),outer sep=0pt]
\node[anchor=east,rectangle,fill=violet!30]
{\strut Corollary~\thethm:~#1};}}%
}%
\mdfsetup{innertopmargin=10pt,linecolor=violet!30,%
linewidth=2pt,topline=true,%
frametitleaboveskip=\dimexpr-\ht\strutbox\relax
}
\begin{mdframed}[]\relax%
\label{#2}}{\end{mdframed}}
%%%%%%%%%%%%%%%%%%%%%%%%%%%%%%
% Warning
\newenvironment{warn}[2][]{%
\refstepcounter{thm}%
\ifstrempty{#1}%
{\mdfsetup{%
frametitle={%
\tikz[baseline=(current bounding box.east),outer sep=0pt]
\node[anchor=east,rectangle,fill=red!30]
{\strut WARNING~\thethm};}}
}%
{\mdfsetup{%
frametitle={%
\tikz[baseline=(current bounding box.east),outer sep=0pt]
\node[anchor=east,rectangle,fill=red!30]
{\strut WARNING~\thethm:~#1};}}%
}%
\mdfsetup{innertopmargin=10pt,linecolor=red!30,%
linewidth=2pt,topline=true,%
frametitleaboveskip=\dimexpr-\ht\strutbox\relax
}
\begin{mdframed}[]\relax%
\label{#2}}{\end{mdframed}}
%%%%%%%%%%%%%%%%%%%%%%%%%%%%%%
% Example
\newenvironment{eg}[2][]{%
\refstepcounter{thm}%
\ifstrempty{#1}%
{\mdfsetup{%
frametitle={%
\tikz[baseline=(current bounding box.east),outer sep=0pt]
\node[anchor=east,rectangle,fill=teal!30]
{\strut Example~\thethm};}}
}%
{\mdfsetup{%
frametitle={%
\tikz[baseline=(current bounding box.east),outer sep=0pt]
\node[anchor=east,rectangle,fill=teal!30]
{\strut Example~\thethm:~#1};}}%
}%
\mdfsetup{innertopmargin=10pt,linecolor=teal!30,%
linewidth=2pt,topline=true,%
frametitleaboveskip=\dimexpr-\ht\strutbox\relax
}
\begin{mdframed}[]\relax%
\label{#2}}{\end{mdframed}}
%%%%%%%%%%%%%%%%%%%%%%%%%%%%%%
% Exercise
\newenvironment{xca}[2][]{%
\refstepcounter{thm}%
\ifstrempty{#1}%
{\mdfsetup{%
frametitle={%
\tikz[baseline=(current bounding box.east),outer sep=0pt]
\node[anchor=east,rectangle,fill=orange!30]
{\strut Exercise~\thethm};}}
}%
{\mdfsetup{%
frametitle={%
\tikz[baseline=(current bounding box.east),outer sep=0pt]
\node[anchor=east,rectangle,fill=orange!30]
{\strut Exercise~\thethm:~#1};}}%
}%
\mdfsetup{innertopmargin=10pt,linecolor=orange!30,%
linewidth=2pt,topline=true,%
frametitleaboveskip=\dimexpr-\ht\strutbox\relax
}
\begin{mdframed}[]\relax%
\label{#2}}{\end{mdframed}}
%%%%%%%%%%%%%%%%%%%%%%%%%%%%%%
%Flawed Proof
\newcounter{flaw}[section]\setcounter{flaw}{0}
\renewcommand{\theflaw}{\arabic{chapter}.\arabic{section}.\arabic{flaw}}
\newenvironment{flaw}[2][]{%
\refstepcounter{flaw}%
\ifstrempty{#1}%
{\mdfsetup{%
frametitle={%
\tikz[baseline=(current bounding box.east),outer sep=0pt]
\node[anchor=east,rectangle,fill=darkgray!30]
{\strut Flawed Proof~\theflaw};}}
}%
{\mdfsetup{%
frametitle={%
\tikz[baseline=(current bounding box.east),outer sep=0pt]
\node[anchor=east,rectangle,fill=darkgray!30]
{\strut Flawed Proof~\theflaw:~#1};}}%
}%
\mdfsetup{innertopmargin=10pt,linecolor=darkgray!30,%
linewidth=2pt,topline=true,%
frametitleaboveskip=\dimexpr-\ht\strutbox\relax
}
\begin{mdframed}[]\relax%
\label{#2}}{\qed\end{mdframed}}

%%%%%%%%%%%%%%%%%%%%%%%%%%%%%%
%Proof
\newcounter{prf}[section]\setcounter{prf}{0}
\renewcommand{\theprf}{\arabic{chapter}.\arabic{section}.\arabic{prf}}
\newenvironment{prf}[2][]{%
\refstepcounter{prf}%
\ifstrempty{#1}%
{\mdfsetup{%
frametitle={%x
\tikz[baseline=(current bounding box.east),outer sep=0pt]
\node[anchor=east,rectangle,fill=cyan!30]
{\strut Proof~\theprf};}}
}%
{\mdfsetup{%
frametitle={%
\tikz[baseline=(current bounding box.east),outer sep=0pt]
\node[anchor=east,rectangle,fill=cyan!30]
{\strut Proof~\theprf:~#1};}}%
}%
\mdfsetup{innertopmargin=10pt,linecolor=cyan!30,%
linewidth=2pt,topline=true,%
frametitleaboveskip=\dimexpr-\ht\strutbox\relax
}
\begin{mdframed}[]\relax%
\label{#2}}{\qed\end{mdframed}}

%%%%%%%%%%%% MACROS %%%%%%%%%%%% MACROS
%%%%%%%%%%%% MACROS %%%%%%%%%%%% MACROS
%% new commands 
\newcommand{\ip}[2]{\langle #1 , #2 \rangle}    %this is to get the correct brackets for inner product
\newcommand{\abs}[1]{\lvert#1\rvert}
\newcommand\diag{\operatorname{diag}}   %%%%%%%%% diag matrix
\newcommand\tr{\mbox{tr}\,}   %%%%%%%%% trace
\newcommand\C{\mathbb C}    %%%%%%%%% the set of complex numbers
\newcommand\R{\mathbb R}    %%%%%%%%% the set of real numbers
\newcommand\Z{\mathbb Z}    %%%%%%%%% the set of integers
\newcommand\N{\mathbb N}
\newcommand\Q{\mathbb Q}
\newcommand\fp{\mathbb F_p} %%% finite field with p elements
\newcommand\fq{\mathbb F_q} %%% finite field with q elements
\newcommand\style{\mathcal}
\newcommand\tran{{}^t} %% Transpose on the left
\newcommand{\eqdef}{\coloneqq} %% Def 
\newcommand{\ds}{\displaystyle}

%% math operators
\DeclareMathOperator{\ran}{Im} %% image
\DeclareMathOperator{\col}{col} %% column space
\DeclareMathOperator{\row}{row} %% row space
\DeclareMathOperator{\spn}{span} %% span
\DeclareMathOperator{\rank}{rank} %% rank
\DeclareMathOperator{\nullsp}{null} %% nullspace
\DeclareMathOperator{\nullity}{nullity} %% nullity
\DeclareMathOperator{\pr}{proj} %projection
\DeclareMathOperator{\ev}{\bf{ev}} %% evaluation
\DeclareMathOperator{\id}{Id} %% identity

\DeclarePairedDelimiter{\ceil}{\lceil}{\rceil}
\DeclarePairedDelimiter{\floor}{\lfloor}{\rfloor}
%% Call as \ceil{x} or \ceil*{x}  to add \left and \right