% Author: Kimberly Golubeva
% Date: 21 July 2020
% Revised: 
% Lauren DeDieu, Jerrod M.~Smith, Kimberly Golubeva and Christian Bagshaw
% A Resource Bank for Writing Intensive Mathematics Courses
% This work is licensed under a  Creative Commons Attribution-NonCommercial-ShareAlike 4.0 International License
% http://creativecommons.org/licenses/by-nc-sa/4.0/
\section{Modular Arithmetic and Divisibility}

\begin{xca}{xca:direct_mod}
Let $p$ be an integer and suppose that $p$ is not divisible by $3.$ Prove that $p^2 \equiv 1 \text{ (mod } 3).$
\end{xca}

\begin{flaw}{flaw:direct_mod} 
Since $3 \nmid p$, this implies that we have two cases. Namely, either \\
$p \equiv 1 \text{ (mod } 3)$ or $p \equiv 2 \text{ (mod } 3)$. \\ 

\noindent \textbf{Case 1:} $p \equiv 1 \text{ (mod } 3)$. By definition, $p = 3k+1$ for some integer $k.$ Then
\begin{align*}
    p^2 &= (3k+1)^2 \\
    &= 9k^2 + 6k +1 \\
    & = 3\left(3k^2+2k + \frac{1}{3}\right)
\end{align*}
I'm not sure where to go from here...\\

\noindent \textbf{Case 2:} $p \equiv 2 \text{ (mod } 3)$. By definition, $p = 3k+2$ for some integer $k.$ Then
\begin{align*}
    p^2 &= (3k+2)^2 \\
    &= 9k^2 + 12k +4 \\
    & = 3\left(3k^2+4k+\frac43\right)
\end{align*}
I'm not sure where to go from here...
\end{flaw}

\clearpage
\subsection{Error classification}


There is only one error
 in the Flawed Proof \ref{flaw:direct_mod}.

 
 \begin{description}
 	\item[C-OS:] Failed to recognize that the first few terms in the expressions $9k^2 + 6k +1$ and $9k^2 + 12k +4$ can be factored independently of the last term, leading to the omission of the final steps in both cases. 
 \end{description}

 
\subsubsection{Error codes}
\begin{itemize}
	\item 	Content Omitted Sections (C-OS)
\end{itemize}
See Section \ref{sec-error} for more information about error classifications.

\clearpage
\subsection{Corrected proof}

The following is a corrected version of Flawed Proof \ref{flaw:direct_mod}.

\begin{prf}{prf:direct_mod}
Since $3 \nmid p$, this implies that we have two cases. Namely, either \\
$p \equiv 1 \text{ (mod } 3)$ or $p \equiv 2 \text{ (mod } 3)$. \\ 

\noindent \textbf{Case 1:} $p \equiv 1 \text{ (mod } 3)$. By definition, $p = 3k+1$ for some integer $k.$ Then
\begin{align*}
    p^2 &= (3k+1)^2 \\
    &= 9k^2 + 6k +1 \\
    & = 3(3k^2+2k)+1 \\
    & = 3l + 1
\end{align*}
where $l \eqdef 3k^2+2k$ is an integer. Thus, $p^2 \equiv 1 \text{ (mod } 3).$ \\

\noindent \textbf{Case 2:} $p \equiv 2 \text{ (mod } 3)$. By definition, $p = 3k+2$ for some integer $k.$ Then
\begin{align*}
    p^2 &= (3k+2)^2 \\
    &= 9k^2 + 12k +4 \\
    &= 9k^2 + 12k +3 + 1 \\
    & = 3(3k^2+4k+1)+1 \\
    & = 3l + 1
\end{align*}
where $l \eqdef 3k^2+4k+1$ is an integer. Thus, $p^2 \equiv 1 \text{ (mod } 3).$

\end{prf}
