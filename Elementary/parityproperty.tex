%Author: Kimberly Golubeva
%Date: July 14 2020
% Lauren DeDieu, Jerrod M.~Smith, Kimberly Golubeva and Christian Bagshaw
% A Resource Bank for Writing Intensive Mathematics Courses
% This work is licensed under a  Creative Commons Attribution-NonCommercial-ShareAlike 4.0 International License
% http://creativecommons.org/licenses/by-nc-sa/4.0/
\clearpage
\section{The Parity Property}

\begin{xca}[]{xca:parityproperty}
Prove that any two consecutive integers have opposite parity. 
\end{xca}

\begin{flaw}{flaw:parityproperty}
Let $n \in \mathbb{Z}$ and suppose that $n$ is even. By definition. $n=2k$ and so $n+1 = 2k+1$, which is odd. Thus, any two consecutive integers have opposite parity. 
\end{flaw}

\clearpage
\subsection{Error classification}

%Provide a brief classification and explanation of the errors in the Flawed Proof \ref{flaw:proof1}. %change the label

There are several errors
% is only one error ... etc.
 in the Flawed Proof \ref{flaw:parityproperty}. 
 
\subsubsection{Error codes}

\begin{description}
 	\item[C-OS:] The case where $n$ is odd was assumed to be true and therefore omitted from the proof.  
 	\item[C-N:] Failed to define what the variable $k$ is. 
 \end{description}

\subsubsection{Error codes}
\begin{itemize}
	\item 	Content Omitted Sections (F-FS)
	\item   Content Notation (C-N)
\end{itemize}
See Section \ref{sec-error} for more information about error classifications.

\clearpage
\subsection{Corrected proof}

The following is a corrected version of Flawed Proof \ref{flaw:parityproperty}. %change the label

\begin{prf}{prf:parityproperty} %change the label
Let $n \in \mathbb{Z}$. We have two cases: 

\begin{enumerate}
    \item \textbf{$n$ is even:} By definition, $n = 2k$ for some integer $k$. Then $n+1 = 2k+1$, which is odd.
    \item \textbf{$n$ is odd:} By definition, $n=2k+1$ for some integer $k$. Then $$n+1 = (2k+1) +1 = 2k+2 = 2(k+1),$$
    which is even since it equals twice some integer $k+1$.
\end{enumerate}
Thus, any two consecutive integers have opposite parity.  
\end{prf}