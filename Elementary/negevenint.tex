% Author: Kimberly Golubeva
% Date: 13 July 2020
% Revised: JMS; 22 January 2021
% Lauren DeDieu, Jerrod M.~Smith, Kimberly Golubeva and Christian Bagshaw
% A Resource Bank for Writing Intensive Mathematics Courses
% This work is licensed under a  Creative Commons Attribution-NonCommercial-ShareAlike 4.0 International License
% http://creativecommons.org/licenses/by-nc-sa/4.0/
\section{Properties of Integers}

\begin{xca}{xca:negevenint}
Prove that the negative of any even integer is even. 
\end{xca}

\begin{flaw}{flaw:negevenint}
Suppose that $n$ is any even integer. By the definition of even, $n=2k$ for some integer $k$. The negative of $n$ is equal to $-n$. Thus, $-n = -2k.$ By dividing both sides of the equation by $-1$, we get back the original equation of $n=2k$. Therefore, the negative of any integer $n$ is an even number. 
\end{flaw}

\clearpage
\subsection{Error classification}


There are several errors
% is only one error ... etc.
 in the Flawed Proof \ref{flaw:negevenint}. 
 
\begin{description}
 	\item[N-N:] Awkward phrasing. Instead, write how the equation $-n=-2k$ was obtained.	
 	\item[C-FI:] The equation $n=2k$ does not inherently imply that $-n$ is even.
 	\item [F-A:] The entire result is asserted by stating that $n=2k$ implies that $-n$ is even.
 \end{description}

 
\subsubsection{Error codes}
\begin{itemize}
	\item 	Novice Notation (N-N)
	\item   Content False Implication (C-FI)
	\item   Fundamental Assertion (F-A)
\end{itemize}
See Section \ref{sec-error} for more information about error classifications.

\clearpage
\subsection{Corrected proof}

The following is a corrected version of Flawed Proof \ref{flaw:negevenint}. 

\begin{prf}{prf:negevenint} 
Suppose that $n$ is an even integer. Then $n=2k$ for some integer $k$. Multiplying both sides by $-1 \in \mathbb{Z}$, we obtain the equation $-n=-2k$. Moreover, we have
\[-n = -2k = 2(-k),\]
where $-k$ is an integer.  Thus, $-n$ is also even. 
\end{prf}
